\documentclass[]{article}

\usepackage{lmodern}


\setlength\parindent{0pt}

\usepackage{amssymb,amsmath}
% for equation numbering:
\renewcommand{\[}{\begin{equation}}
\renewcommand{\]}{\end{equation}}

\newcommand{\R}{\textsf{R}}

\usepackage{ifxetex,ifluatex}
\usepackage{fixltx2e} % provides \textsubscript
\ifnum 0\ifxetex 1\fi\ifluatex 1\fi=0 % if pdftex
  \usepackage[T1]{fontenc}
  \usepackage[utf8]{inputenc}
\else % if luatex or xelatex
  \ifxetex
    \usepackage{mathspec}
    \usepackage{xltxtra,xunicode}
  \else
    \usepackage{fontspec}
  \fi
  \defaultfontfeatures{Mapping=tex-text,Scale=MatchLowercase}
  \newcommand{\euro}{€}
\fi
% use upquote if available, for straight quotes in verbatim environments
\IfFileExists{upquote.sty}{\usepackage{upquote}}{}
% use microtype if available
\IfFileExists{microtype.sty}{\usepackage{microtype}}{}
\usepackage[left=3.5cm,right=3.5cm,top=2cm,bottom=4cm]{geometry}






\usepackage{longtable,booktabs, dcolumn, rotating}
\usepackage{subfig}

\ifxetex
  \usepackage[setpagesize=false, % page size defined by xetex
              unicode=false, % unicode breaks when used with xetex
              xetex]{hyperref}
\else
  \usepackage[unicode=true]{hyperref}
\fi
\hypersetup{breaklinks=true,
            bookmarks=true,
            pdfauthor={David Benček},
            pdftitle={Einführung in die statistische Analyse mit},
            colorlinks=true,
            citecolor=blue,
            urlcolor=blue,
            linkcolor=magenta,
            pdfborder={0 0 0}}
\urlstyle{same}  % don't use monospace font for urls



%\setlength{\parindent}{0pt}
%\setlength{\parskip}{6pt plus 2pt minus 1pt}
%\setlength{\emergencystretch}{3em}  % prevent overfull lines

\setcounter{secnumdepth}{0}



\title{Einführung in die statistische Analyse mit \R{}}
\author{David Benček}
\date{12. August 2015}


% Custom header for syllabus

\makeatletter
\renewcommand\maketitle{%
 \null\hfill Stand: \@date\vspace{0.5cm}
 {\bfseries
  \begin{center}
      {\Large \@title}\\
      {{\mdseries WS 2015/16}}
  \end{center}
  \begin{minipage}[t][][t]{0.23\linewidth}
			Termin:\\
			Raum:\\
			Dozent:\\
			E-Mail:\\
			Telefon:\\
			Sprechstunde:\\
		\end{minipage}
		\begin{minipage}[t][][t]{0.7\linewidth}
			Montag, 10:00h--12:00h\\
			WSP1 R.114\\
			\@author\\
			\href{mailto:david.bencek@ifw-kiel.de}{\nolinkurl{david.bencek@ifw-kiel.de}}\\
			0431-8814-470\\
			nach Vereinbarung\\
		\end{minipage}
 }
 
}
\makeatother

% Change size of section titles
\usepackage{titlesec}
\titleformat*{\section}{\large\bfseries}
\titleformat*{\subsection}{\bfseries}


\begin{document}
\maketitle






\section{Kursbeschreibung}\label{kursbeschreibung}

Der Kurs vermittelt ein Grundwissen der statistischen Analyse. Dabei
steht die praktische Anwendung in Form der Datenanalyse im Vordergrund.
Während des gesamten Kurses wird hierfür die freie Programmiersprache
\R{} verwendet. Es werden weder Programmier- noch Statistikkenntnisse
vorausgesetzt.

Ziel des Kurses ist, die TeilnehmerInnen in die Lage zu versetzen, eine
explorative Datenanalyse in \R{} durchzuführen sowie einfache
Regressionsmodelle nachzuvollziehen, selbst zu berechnen und zu
interpretieren.

\section{Kursplan (vorläufig)}\label{kursplan-vorlaufig}

\begin{longtable}[c]{@{}rll@{}}
\toprule
\begin{minipage}[b]{0.18\columnwidth}\raggedleft\strut
Sitzung
\strut\end{minipage} &
\begin{minipage}[b]{0.18\columnwidth}\raggedright\strut
Datum
\strut\end{minipage} &
\begin{minipage}[b]{0.42\columnwidth}\raggedright\strut
Thema
\strut\end{minipage}\tabularnewline
\midrule
\endhead
\begin{minipage}[t]{0.18\columnwidth}\raggedleft\strut
1
\strut\end{minipage} &
\begin{minipage}[t]{0.18\columnwidth}\raggedright\strut
19.10.15
\strut\end{minipage} &
\begin{minipage}[t]{0.42\columnwidth}\raggedright\strut
Ziele quantitativer Forschung, Grundbegriffe der Datenanalyse
\strut\end{minipage}\tabularnewline
\begin{minipage}[t]{0.18\columnwidth}\raggedleft\strut
2
\strut\end{minipage} &
\begin{minipage}[t]{0.18\columnwidth}\raggedright\strut
26.10.15
\strut\end{minipage} &
\begin{minipage}[t]{0.42\columnwidth}\raggedright\strut
Einführung in \R{} \& RStudio, grundlegende Funktionen
\strut\end{minipage}\tabularnewline
\begin{minipage}[t]{0.18\columnwidth}\raggedleft\strut
3
\strut\end{minipage} &
\begin{minipage}[t]{0.18\columnwidth}\raggedright\strut
02.11.15
\strut\end{minipage} &
\begin{minipage}[t]{0.42\columnwidth}\raggedright\strut
Statistische Grundlagen, Berechnung in \R{}
\strut\end{minipage}\tabularnewline
\begin{minipage}[t]{0.18\columnwidth}\raggedleft\strut
4
\strut\end{minipage} &
\begin{minipage}[t]{0.18\columnwidth}\raggedright\strut
09.11.15
\strut\end{minipage} &
\begin{minipage}[t]{0.42\columnwidth}\raggedright\strut
Statistische Grundlagen, Berechnung in \R{}
\strut\end{minipage}\tabularnewline
\begin{minipage}[t]{0.18\columnwidth}\raggedleft\strut
5
\strut\end{minipage} &
\begin{minipage}[t]{0.18\columnwidth}\raggedright\strut
16.11.15
\strut\end{minipage} &
\begin{minipage}[t]{0.42\columnwidth}\raggedright\strut
Deskriptive Statistiken und Datenvisualisierung
\strut\end{minipage}\tabularnewline
\begin{minipage}[t]{0.18\columnwidth}\raggedleft\strut
6
\strut\end{minipage} &
\begin{minipage}[t]{0.18\columnwidth}\raggedright\strut
23.11.15
\strut\end{minipage} &
\begin{minipage}[t]{0.42\columnwidth}\raggedright\strut
Plots und Datenverarbeitung
\strut\end{minipage}\tabularnewline
\begin{minipage}[t]{0.18\columnwidth}\raggedleft\strut
7
\strut\end{minipage} &
\begin{minipage}[t]{0.18\columnwidth}\raggedright\strut
30.11.15
\strut\end{minipage} &
\begin{minipage}[t]{0.42\columnwidth}\raggedright\strut
Datenverarbeitung
\strut\end{minipage}\tabularnewline
\begin{minipage}[t]{0.18\columnwidth}\raggedleft\strut
8
\strut\end{minipage} &
\begin{minipage}[t]{0.18\columnwidth}\raggedright\strut
07.12.15
\strut\end{minipage} &
\begin{minipage}[t]{0.42\columnwidth}\raggedright\strut
Lineare Regression
\strut\end{minipage}\tabularnewline
\begin{minipage}[t]{0.18\columnwidth}\raggedleft\strut
9
\strut\end{minipage} &
\begin{minipage}[t]{0.18\columnwidth}\raggedright\strut
11.01.16
\strut\end{minipage} &
\begin{minipage}[t]{0.42\columnwidth}\raggedright\strut
Logit-Modell
\strut\end{minipage}\tabularnewline
\begin{minipage}[t]{0.18\columnwidth}\raggedleft\strut
10
\strut\end{minipage} &
\begin{minipage}[t]{0.18\columnwidth}\raggedright\strut
18.01.16
\strut\end{minipage} &
\begin{minipage}[t]{0.42\columnwidth}\raggedright\strut
Zähl-Modell
\strut\end{minipage}\tabularnewline
\begin{minipage}[t]{0.18\columnwidth}\raggedleft\strut
11
\strut\end{minipage} &
\begin{minipage}[t]{0.18\columnwidth}\raggedright\strut
25.01.16
\strut\end{minipage} &
\begin{minipage}[t]{0.42\columnwidth}\raggedright\strut
Anwendungsbeispiele
\strut\end{minipage}\tabularnewline
\begin{minipage}[t]{0.18\columnwidth}\raggedleft\strut
12
\strut\end{minipage} &
\begin{minipage}[t]{0.18\columnwidth}\raggedright\strut
01.02.16
\strut\end{minipage} &
\begin{minipage}[t]{0.42\columnwidth}\raggedright\strut
Wiederholung/Fragestunde
\strut\end{minipage}\tabularnewline
\begin{minipage}[t]{0.18\columnwidth}\raggedleft\strut
13
\strut\end{minipage} &
\begin{minipage}[t]{0.18\columnwidth}\raggedright\strut
08.02.16
\strut\end{minipage} &
\begin{minipage}[t]{0.42\columnwidth}\raggedright\strut
Klausur: Replikation einer Studie
\strut\end{minipage}\tabularnewline
\bottomrule
\end{longtable}

\section{Leistungsnachweis}\label{leistungsnachweis}

KursteilnehmerInnen werden zwei Leistungsnachweise erbringen:

\begin{enumerate}
\def\labelenumi{\arabic{enumi}.}
\itemsep1pt\parskip0pt\parsep0pt
\item
  Aufgabenblatt mit Problemstellungen. Zu lösen über die
  Weihnachtsferien, Abgabefrist: 10.01.16. (Gewicht: 25\%)
\item
  Replikation einer veröffentlichten Studie. Datenanalyse als Klausur.
  (Gewicht: 75\%)
\end{enumerate}

Die Gesamtnote für den Kurs ergibt sich als gewichtetes Mittel beider
Teile.

\section{Literatur}\label{literatur}

Der Kurs baut auf Teilen mehrerer Lehrbücher auf, deshalb sind die
aufgelisteten Werke als Vorschläge für das Selbststudium zu verstehen.
Notwendiges Material wird über OLAT bereitgestellt. Darüber hinaus
bietet \R{} eine große und sehr aktive Community, durch die zahllose
Anleitungen und Hilfestellungen online verfügbar sind.

\begin{itemize}
\item
  Nicole Radziwill (2015): Statistics (The Easier Way) With R.
  \href{http://www.amazon.de/Statistics-Easier-Way-informal-statistics/dp/0692339426/}{{[}Amazon{]}}
\item
  Andy Field, Jeremy Miles, Zoe Field (2012): Discovering Statistics
  Using R.
  \href{http://www.amazon.de/gp/product/1446200469/}{{[}Amazon{]}}
\item
  Peter Dalgaard (2008): Introductory Statistics with R.
  \href{http://www.amazon.de/Introductory-Statistics-R-Computing/dp/0387790535/}{{[}Amazon{]}}
\end{itemize}

\section{Software}\label{software}

Im Kurs wird das freie Statistikprogramm \R{} verwendet. Es kann unter
\href{https://www.r-project.org/}{\url{https://www.r-project.org/}}
heruntergeladen und installiert werden. Außerdem kommt zur bequemeren
Anwendung
\href{https://www.rstudio.com/products/rstudio/download/}{R-Studio} als
Benutzeroberfläche zum Einsatz. Beide Programme sind für Windows, Mac
und Linux verfügbar.




\end{document}